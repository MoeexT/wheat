
\subsection{\textbf{研究背景及意义}}
小麦是我国三大粮食作物之一,其种植区域主要分布在长江以北的大部分地区,种植面积位居第二,仅次于水稻。
病害是影响小麦等农作物产量和质量的首要问题。全世界范围内小麦病害大约有200多种,每年造成的产量
损失约为15\%\textasciitilde20\%。在我国危害较严重的有三十多种,其中以白粉病(Wheat powdery mildew)、
锈病(Puccinia striiformis West,包括条锈、杆锈、叶锈)、叶枯病(Wheat leaf blotch)、
赤霉病等在我国主要小麦产地分布较广,为害较为严重\cite{CGRIS}。

传统形式的小麦病虫害是依靠经验识别、人工喷施农药进行防治的。在大面积的小麦种植模式下,人工防
治不仅需要大量的人力物力,而且在喷施农药的过程中极为不安全。更进一步地讲,即使是在技术人员的
帮助下进行病害防治,也不能做到实时监控小麦病害情况、及时实施防治工作。因此如何做到实时监控并
报告小麦病害情况成为了现代农业生产中的重要目标。

自1980年机器学习被称为一个独立的方向开始,经过一代又一代人的努力,诞生出了大量经典的分类算法。
其中以朴素贝叶斯(Naive Bayes, NB)、Logistic回归(Logistic Regression, LR)、决策树
(Decision Tree, DT)、支持向量机(Support Vector Machine, SVM)等浅层机器学习模型最常用,
它们的出现为小麦病害的自动识别提供了有力的理论支持。但是这些经典分类算法的图像特征提取策略是基
于先验知识制定的,效率不高且不适合应用在大规模特征提取方面\cite{article1}。 近年来,由AlphaGo
带来的人工智能热潮使得深度学习一词出现在公共视野里。 深度学习是机器学习领域中一种以人工神经网络
为架构,通过数据进行表征学习的算法。

如此一来,将现代的深度学习技术与传统图像处理相结合的方式成为了农作物病害识别的新手段。深度学习在
图像处理领域的优势不仅仅在于能够准确地提取特征,还在于它通过处理大量图像数据时能不断地自我学习并
取得更高的准确率。

在以深度学习为主要手段的图像处理过程中,卷积神经网络模型对图像的处理由很高的优势,故本文研究的内
容会以卷积神经网络为基础,通过建立不同结构的神经网络模型并加以比较,找出最适合小麦病害识别的深度
网络模型。
