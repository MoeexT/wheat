\begin{abstract}
    为了实时监控小麦病害情况并及时采取防治措施,找出一种适合分类处理小麦常见病害的神经网络模型。
    本文首先以小麦病害图片资料为基础,经过挑选、裁剪后对图像进行序列化操作形成数据集,
    然后通过构建的多个深度神经网络模型进行学习,在学习过程中使用随机梯度下降法对学习过程进行控制。
    然后在此基础上改进表现较差的神经网络模型,在两个维度上形成对比,以此寻找最适合处理
    小麦病害分类问题的深度神经网络模型。实验结果表明,在参与实验的多个神经网络结构中,
    以卷积神经网络(convolutional neural networks,CNN)表现最为出众,
    整体识别准确率达99\%,(交叉)验证准确率在(validation accuracy)75\%左右。
    这表明使用卷积神经网络进行小麦常见病害的识别是有效且可行的,理论上为小麦病害的实时分析
    提供了有效且强力的分析手段。

    \leftline{\textbf{关键字 } \@ 小麦病害;卷积神经网络;图像识别;特征分析;深度学习}
\end{abstract}
\newpage

\begin{center}
    \fontsize{18pt}{18pt}\textbf{Feature analysis of wheat disease classification model based on deep learning}
\end{center}
\begin{enabstract}
    In order to monitor the disease situation of wheat in real time and take timely preventive measures, find a suitable wheat for classification.
    See the neural network model of the disease. This paper firstly based on the picture data of wheat diseases, after selection and cutting,
    The image is serialized to form a data set, and then learned by constructing multiple deep neural network models.
    The learning process is controlled using a stochastic gradient descent method during the learning process. Then improve performance on this basis
    Poor neural network model, contrasting in two dimensions to find the most suitable classification for wheat disease
    The deep neural network model of the problem. The experimental results show that convolution is performed in multiple neural network structures participating in the experiment.
    Convergent neural networks (CNN) perform most outstandingly, and the overall recognition accuracy rate is up to
    99\%, (cross) verification accuracy is about 75\% (validation accuracy). This indicates the use of convolutional neural networks
    It is effective and feasible to identify the common diseases of wheat. In theory, it provides a real-time analysis of wheat diseases.
    Effective and powerful analytical tools.

    \leftline{\textbf{Keywords } \@ Wheat Disease;\ CNN;\ Image Identification;\ Characteristics;\ Deep Learning}
\end{enabstract}
\newpage