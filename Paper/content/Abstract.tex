\begin{abstract}
    为了实时监控小麦病害情况并及时采取防治措施,构建了一个适合处理小麦常见病害的卷积网络分类模型。
    本文以小麦病害图片为基础,经过挑选、裁剪、图像矩阵序列化等操作获得标准数据集,
    通过构建多个深度神经网络模型进行对比,并使用随机梯度下降法对模型权重进行学习。
    在此基础上改进表现较差的神经网络模型,以说明本文卷积网络的处理能力。
    实验结果表明,本文提出的卷积神经网络模型在小麦病害分类问题上表现良好,
    验证准确率在(Validation Accuracy)93\%左右。在和经典卷积模型LeNet-5对比实验后发现,
    本文模型准确率高于LeNet-5(84\%)。
    这表明使用卷积神经网络进行小麦常见病害的识别是有效且可行的,理论上为小麦病害的实时分析
    提供了有效且强力的分析手段。

    {\textbf{关键词 } \@ 小麦病害;卷积神经网络;图像识别;特征分析;深度学习}
\end{abstract}
\newpage

\begin{center}
    \fontsize{18pt}{18pt}\textbf{Feature analysis of wheat disease classification model based on deep learning}
\end{center}
\begin{enabstract}
    In order to monitor the disease situation of wheat in real time and take preventive measures in time, 
    a convolutional network classification model suitable for dealing with common diseases of wheat was constructed. 
    Based on the pictures of wheat diseases, this paper obtains standard datasets through selection, 
    cropping, image matrix serialization, etc., constructs multiple deep neural network models for comparison, 
    and uses stochastic gradient descent method to learn model weights. On this basis, the poor performance 
    neural network model is improved to illustrate the processing power of the convolutional network. 
    The experimental results show that the proposed convolutional neural network model performs well on 
    the classification of wheat diseases, and the verification accuracy rate is about 93\% 
    (Validation Accuracy). After comparing with the classical convolution model LeNet-5, 
    the accuracy of the model is higher than that of LeNet-5 (84\%). 
    This indicates that the use of convolutional neural networks for the identification of 
    common wheat diseases is effective and feasible, and theoretically provides an effective and 
    powerful analytical tool for real-time analysis of wheat diseases.

    {\textbf{Keywords } \@ Wheat Disease;\ CNN;\ Image Identification;\ Characteristics;\ Deep Learning}
\end{enabstract}
\newpage