\subsection{\hei\xiaosan\textbf{农作物病害识别的国内外研究现状}}
  \subsubsection{\hei\sihao\textbf{国内研究现状}}
    国内将深度学习应用在图像处理尤其是农作物病害识别方面的研究起步较晚,大多数
    在机器学习背景下开始此领域研究的,近年来该领域的研究呈现井喷式发展,并
    取得了许多卓有成效的识别技术。
    2009年,王守志等(王守志等,2009)实现了基于核K-均值聚类方法的玉米叶部病害识别,
    实验涉及的4种玉米病害识别准确率达82.5\% \cite{王守志2009基于核};
    2011年,陈丽等(陈丽等,2011)提出了一种基于图像处理技术和概率神经网络技术的玉米
    叶部病害识别方法,利用遗传算法优化选择出4个分类能力强的分类特征,由概率网络(PNN)
    分类器识别病害,平均识别准确率为90.4\%,高于BP神经网络\cite{陈丽2011概率神经网络在玉米叶部病害识别中的应用};
    2012年,张建华等(张建华等,2012)提出了一种在自然条件下基于粗糙集和BP神经网络的
    棉花病害识别方法,准确识别了4种棉花病害,平均识别准确率达到92.72\% \cite{张建华2012基于粗糙集和};
    王树文等(王树文等,2012)利用基本图像处理方法对黄瓜叶部病害图像进行处理,综合运用
    二次分割、形态学滤波得到病斑区域。提取三种特征并采用BP算法训练多层前向人工神经网络
    对黄瓜病害进行分类,检测系统的黄瓜叶部病害平均识别精度为95.31\% \cite{王树文2012基于图像处理技术的黄瓜叶片病害识别诊断系统研究};
    2013年,张飞云等(张飞云等,2013)利用量子神经网络进行玉米病害分类识别,对玉米灰斑病、
    玉米普通锈病和玉米小斑病的识别准确率达92.5\%、97.5\%和92.5\%,高于误差反向传播神经网
    络法的识别率(分别为90.0\%、90.0\%和92.5\%),可用于玉米叶部病害识别\cite{张飞云2013基于量子神经网络和组合特征参数的玉米叶部病害识别};
    2014年,余秀丽等(余秀丽等,2014)设计并实现了一种基于SVM(Support Vector Machine,
    支持向量机)的小麦叶部常见病害识别方法。随机试验结果表明,利用所提取的特征可以有效地
    实现小麦叶部常见病害的识别,基于形状特征综合识别率可达99.33\%,利用支持向量机算法进
    行小麦病害叶片识别是有效的、可行的\cite{余秀丽2014基于};
    2015年,谭文学等(谭文学等,2015)设计了深度学习神经网络的果蔬果体图像识别方法,基于
    对网络误差的传播分析,提出弹性动量的参数学习方法,以苹果为例进行果体病理图像的识别试验。
    结果表明,该方法召回率为98.4\% \cite{谭文学2015基于弹性动量深度学习神经网络的果体病理图像识别};
    2018年,张航等(张航等,2018)提出了一种基于卷积神经网络的小麦病害识别方法,构建一个
    具有五层结构的深度学习模型并利用随机梯度下降法进行学习过程控制,其综合识别率可达99\% \cite{张航2018一种基于卷积神经网络的小麦病害识别方法}。

  \subsubsection{\hei\sihao\textbf{国外研究现状}}
    国外关于农作物病害识别的研究起步较早,在上世纪九十年代已有多种识别方法被提出。
    在早期的处理方法大多是对病害图像进行前期处理,包括图像分割、滤波、简单分类等技术,
    或是辅助以人工分析,然后再结合农作物病理学知识对处理后的数据进行分类识别。
    1997年,Tucker与Chakraborty(Tucker et al.,1997)提供了一种专用软件,可以检测
    向日葵和燕麦叶片上的病变,提供病变数量和类型以及病害严重程度的数据,但是该软件的分类
    准确率达不到预想程度 \cite{tucker1997quantitative};
    1999年,Sasaki等(Sasaki et al.,1999)使用光谱反射特性和滤光片图像构建了一个植物
    病害自动诊断系统,发现500,600和650mm的滤光镜图像比其他滤光镜图像更适合于识别,在
    创建鉴别参数对健康叶片和患病叶片进行分类后,达到了5\%或更小的误差率\cite{sasaki1999automatic}。

    由于图像处理技术的限制,早期的农作物病害识别技术并不能满足人们的要求。近年来,随着
    图像处理和人工智能的发展,机器学习应用到农作物病害识别领域的例子越来越多。
    2008年,Phadikar和Sil等(Phadikar et al.,2008)介绍了一种基于水稻植株感染图像的
    水稻病害检测系统,该系统使用图像生长、分割技术处理感染部分,然后使用SOM神经网络将四种
    叶子的感染部分进行分类处理,实验结果令人满意\cite{phadikar2008rice};
    2014年,Mathura与Uttar Pradesh(Mathura et al.,2014)介绍了一种基于邻近像素点像素
    强度的改进和差直方图,与梯度滤波器相配合使用可以对苹果病害的检测达到99\%的准确率 \cite{dubey2014fruit};
    2016年,Sladojevic等(Sladojevic et al.,2016)实现了最新一代的卷积神经网络(CNN),
    该模型能够识别健康叶片中的13种不同类型的植物病害,实验结果的平均精确度达到了96.3\% \cite{sladojevic2016deep};
    2017年,Fuentes和Yoon等(Fuentes et al.,2017)提出了一种深度学习方法来检测番茄病害,
    该方法含有三种神经元架构:基于区域的快速卷积网络(Faster R-CNN)、基于区域的全卷积网络
    (R-FCN)和单发多核检测器(SSD),实验结果表明,该系统能有效识别九种不同类型的病虫害\cite{fuentes2017robust};

  \subsection{\hei\xiaosan\textbf{深度学习与神经网络的发展}}
  深度学习是机器学习领域中一种以人工神经网络为架构,根据数据进行表征学习的算法。同时它也是
  神经网络、最优化方法、模式识别、人工智能等领域的交叉学科,主要解决了贡献度的分配问题。

  对于深度学习的研究最早可以追溯到1943年,由神经科学家麦卡洛克(W.S.McCulloch)和数学家皮
  兹(W.Pitts)建立了神经网络和数学模型,称为MCP模型。该模型是按照生物神经元的结构和工作
  原理构造出来的一个抽象的简化模型,所谓的“模拟大脑”由此诞生,深度学习和人工神经网络的大门
  由此开启\cite{mcculloch1990logical}。1958年,计算机科学家罗森布拉特(Rosenblatt)提出
  了两层神经元组成的神经网络,称之为“感知器”(Perceptrons)。第一次将MCP模型用于机器学习分
  类。“感知器”算法使用MCP模型对输入的多维数据进行二分类,且能够使用梯度下降法从训练样本中自
  动学习更新权值。1962年,该方法被证明能够收敛,理论与实践效果引起第一次神经网络浪潮。

  神经网络之父Geoffery Hinton在1986年发明了适用于多层感知器(MLP)的BP(Backpropagation)
  算法,并采用Sigmoid进行非线性映射,有效解决了非线性分类和学习的问题。引起了神经网络
  的第二次热潮。Sigmoid 函数是一个在生物学中常见的S型的函数,也称为S型生长曲线。在信息科学
  中,由于其单增以及反函数单增等性质,Sigmoid函数常被用作神经网络的阈值函数,将变量映射到0,
  1之间。90年代中期,支持向量机(Support Vector Machine,SVM)算法诞生等各种浅层机器学习
  模型被提出,SVM也是一种有监督的学习模型,应用于模式识别,分类以及回归分析等。支持向量机以
  统计学为基础,和神经网络有明显的差异,支持向量机等算法的提出再次阻碍了深度学习的发展。2006
  年Geoffrey Hinton和他的学生Ruslan Salakhutdinov提出了深层网络训练中梯度消失问题的解决方
  案:无监督预训练对权值进行初始化+有监督训练微调\cite{hinton2006reducing}。2011年,ReLU
  激活函数被提出,该激活函数能够有效的抑制梯度消失问题。

  2011年以来,微软首次将深度学习应用在语音识别上,取得了重大突破。微软研究院和Google的语音识别
  研究人员先后采用DNN技术降低语音识别错误率20%~30%,是语音识别领域十多年来最大的突破性进展。
  2012年,Hinton课题组为了证明深度学习的潜力,首次参加ImageNet图像识别比赛,其构建的CNN网络
  AlexNet在ImageNet评测上将错误率从26\%降低到15\%,在第二名(SVM方法)面前以压倒性的优势夺得
  冠军。也正是由于该比赛,CNN吸引到了众多研究者的注意。AlexNet的创新点在于:
  
  \begin{enumerate}[(1)]
    \dawu\item 首次采用ReLU激活函数,极大增大收敛速度且从根本上解决了梯度消失问题。
    \dawu\item 由于ReLU方法可以很好抑制梯度消失问题,AlexNet抛弃了“预训练+微调”的方法,完全采用
      有监督训练。也正因为如此,DL的主流学习方法也因此变为了纯粹的有监督学习。
    \dawu\item 扩展了LeNet5结构,添加Dropout层减小过拟合,LRN层增强泛化能力/减小过拟合。
    \dawu\item 第一次使用GPU加速模型计算。
  \end{enumerate}

  2016年3月,由谷歌(Google)旗下DeepMind公司开发的AlphaGo(基于深度学习)与围棋世界冠军、职业
  九段棋手李世石进行围棋人机大战,以4比1的总比分获胜;2016年末2017年初,该程序在中国棋类网站上
  以“大师”(Master)为注册帐号与中日韩数十位围棋高手进行快棋对决,连续60局无一败绩;2017年5月,
  在中国乌镇围棋峰会上,它与排名世界第一的世界围棋冠军柯洁对战,以3比0的总比分获胜。围棋界公认
  阿尔法围棋的棋力已经超过人类职业围棋顶尖水平。

  深度学习虽然已经成为众多科研领域的热门研究内容,但是目前还处于发展阶段,不管是理论方面还是实践
  方面都还有许多问题待解决,不过由于我们处在了一个“大数据”时代,以及计算资源的大大提升,新模型、
  新理论的验证周期会大大缩短,人工智能的开启必然会很大程度地改变这个世界。
