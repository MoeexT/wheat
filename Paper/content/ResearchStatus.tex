\subsection{\hei\xiaosan\textbf{农作物病害识别的国内外研究现状}}
  \subsubsection{\hei\sihao\textbf{国内研究现状}}
    国内在农作物病害识别方面的研究起步较晚,多数是在国外相关研究启发下开始的。
    近年来机器学习领域的研究呈现井喷式发展,随之带动了农作物病害识别的研究,
    并取得了许多卓有成效的识别技术。
    2009年,王守志等实现了基于核K-均值聚类方法的玉米叶部病害识别,实验涉及的4
    种玉米病害识别准确率达82.5\% \cite{王守志2009基于核};
    2011年,陈丽等提出了一种基于图像处理技术和概率神经网络技术的玉米叶部病害识
    别方法,利用遗传算法优化选择出4个分类能力强的分类特征,由概率网络(PNN)分
    类器识别病害,平均准确率为90.4\%,高于BP\zs 神经网络\cite{陈丽2011概率神经网络在玉米叶部病害识别中的应用};
    2012年,张建华等提出了一种基于粗糙集和BP神经网络的棉花病害识别方法,
    该方法能准确识别4种棉花病害,平均识别准确率达到92.72\% \cite{张建华2012基于粗糙集和};
    王树文\zs 等利用基本图像处理\zs 方法对黄瓜叶部病害\zs 图像进行处理,\zs 综合运用\zs 二次分割、
    形态学滤波得到病斑区域。提取三种特征并采用BP算法训练多层前向神经网络对
    黄瓜病害进行分类,该系统的平均识别准确度为95.31\% \cite{王树文2012基于图像处理技术的黄瓜叶片病害识别诊断系统研究};
    2013年,张飞云等利用\zs 量子神经网络进行\zs 玉米病害分类识别,对玉米灰斑病、玉米普
    通锈病和玉米小斑病的识别准确率达92.5\%、97.5\%和92.5\%,高于误差反向传播
    神经网络法的识别率(分别为90.0\%、90.0\%和92.5\%),可用于玉米叶部病害识
    别\cite{张飞云2013基于量子神经网络和组合特征参数的玉米叶部病害识别};
    2014年,余秀丽等设计并实现了一种基于\zs SVM(Support Vector Machine,\zs 支持向量机)
    的小麦\zs 叶部常见病害识别\zs 方法。随机试验结果\zs 表明,利用所提取\zs 的特征可以有效地实\zs 
    现小麦叶部常见病害的\zs 识别,基于形状\zs 特征综合识别率可达\zs 99.33\%,利用支持向量
    机算法\zs 进行小麦病害叶片识别是\zs 有效的、可行的\cite{余秀丽2014基于};
    2018年,张航等提出了\zs 一种基于卷积神经\zs 网络的小麦病害识别方法,\zs 利用随机梯度
    下降法\zs 对一个具有五层结构\zs 的网络模型行学习过程控制,\zs 其综合识别率可达99\% 
    \cite{张航2018一种基于卷积神经网络的小麦病害识别方法}。

  \subsubsection{\hei\sihao\textbf{国外研究现状}}
    国外关于农作物病害识别的研究起步较早,在上世纪九十年代已有多种识别方法被提出。
    在早期的处理方法大多是对病害图像进行前期处理,包括图像分割、滤波、简单分类等
    技术,或是辅助以人工分析,然后再结合农作物病理学知识对处理后的数据进行分类识别。
    1997年,Tucker与Chakraborty提供了一种专用软件,可以检测向日葵和燕麦叶片上的病
    变,提供病变数量和类型以及病害严重程度的数据,但是该软件的分类准确率达不到预想
    程度 \cite{tucker1997quantitative};
    1999年,Sasaki等使用光谱反射特性和滤光片图像构建了一个植物病害自动诊断系统,
    发现500、600和650mm的滤光镜图像比其他滤光镜图像更适合于识别,在创建鉴别参数对
    健康叶片和患病叶片进行分类后,达到了5\%或更小的误差率\cite{sasaki1999automatic}。

    由于图像处理技术的限制,早期的农作物病害识别技术并不能满足人们的要求。近年来,
    随着图像处理和人工智能的发展,机器学习应用到农作物病害识别领域的例子越来越多。
    2008年,Phadikar和Sil等介绍了一种基于水稻植株感染图像的水稻病害检测系统,该
    系统使用图像生长、分割技术处理感染部分,然后使用SOM神经网络将四种叶子的感染部
    分进行分类处理,实验结果令人满意\cite{phadikar2008rice};
    2014年,Mathura与Uttar Pradesh介绍了一种基于邻近像素点像素强度的改进和差直方
    图,与梯度滤波器相配合使用可以对苹果病害的检测达到99\%的准确率 \cite{dubey2014fruit};
    2016年,Sladojevic等实现了最新一代的卷积神经网络(CNN),该模型能够识别健康
    叶片中的13种不同类型的植物病害,实验结果的平均精确度达到了96.3\% \cite{sladojevic2016deep};
    2017年,Fuentes和Yoon等提出了一种深度学习方法来检测番茄病害,该方法含有三种
    神经元架构:基于区域的快速卷积网络(Faster R-CNN)、基于区域的全卷积网络
    (R-FCN)和单发多核检测器(SSD),实验结果表明,该系统能有效识别九种不同类型
    的病虫害\cite{fuentes2017robust};

\subsection{\hei\xiaosan\textbf{深度学习与神经网络的发展}}
  深度学习是机器学习领域中一种以人工神经网络为架构,根据数据进行表征学习的算法。
  它的前身是人工神经网络,基本特点是模仿人脑神经元处理和传递信息的方式,本质是解决贡献度的分配问题。

  对深度学习的研究\zs 最早可以追溯到1943年,\zs 神经科学家麦卡洛克(W.S.McCulloch)\zs 和数学家皮
  兹(W.Pitts)建立了一个\zs 基于神经网络和数学\zs 的模型,称为MCP模型。MCP模型\zs 是按照大脑神经元的\zs 结构和工作
  原理构造出来的\zs 一个简化了的、抽象的\zs 模型,也就是所谓的“模拟\zs 大脑”,深度学习和\zs 人工神经网络的大门
  由此开启\cite{mcculloch1990logical}。1958\zs 年,计算机科学\zs 家罗森布拉特(Rosenblatt)\zs 提出
  了一个由\zs 两层神经元组成的\zs 神经网络,称之为“感\zs 知器”(Perceptrons)。这也是第一次将MCP\zs 模型用于机器学习分
  类。“感知器”算法使用MCP\zs 模型对输入的多维\zs 数据分为两类,且能够使用\zs 梯度下降法在训练\zs 过程中中自
  动学习并\zs 更新权值。1962年,Novikoff定理证明该方法具有收敛性,理论与实践效果引起第一次神经网络热潮。

  神经网络之父\zs Geoffery Hinton\zs 于19\zs 86年发明了适用于多层\zs 感知器(MLP)的BP(Backpropagation)
  算法,该方法采用Sig\zs moid进行非线\zs 性映射,有效解决了非线\zs 性分类和学习\zs 的问题。
  Sigmoid函数的输出在(0,1)内,单调连续且易于求导,非常适合用作输出层。
  但是它也有软\zs 饱和性的缺点,一旦落入\zs 饱和区,$f'(x)$就会变得接近于0,很容易产生梯度消失。
  90年代中期,支持\zs 向量机(Support Vec\zs tor Machine,SVM)算法诞生,随之\zs 各种浅层机器学习\zs 
  模型被提出。SVM\zs 是一种有监督\zs 的学习模型,通常应用于模式\zs 识别,分类以及回归\zs 分析等。支持向量\zs 机以
  统计学为基础,和\zs 神经网络有明\zs 显的差异,支持向量机等算法的提出再次使深度学习的发展受到阻碍。2006
  年Geoffrey Hinton和Ruslan Salakhutdinov提出了深层网络训练中梯度消失的解决方案:
  无监督预训练下对权值进行初始化,有监督训练权值微调\cite{hinton2006reducing}。2011年,ReLU
  激活函数被提出,该激活函数能够有效的抑制梯度消失问题。

  2011年,微软\zs 首次将深度\zs 学习应用在语音识别\zs 领域,取得了重大\zs 突破。微软研\zs 究院和Goo\zs gle的语音识别
  研究\zs 人员先后采用卷积\zs 神经网络使语音\zs 识别错误率降低20%\textasciitilde30%,是语音\zs 识别领域十多\zs 年来的突破性进展。
  20\zs 12年,Hin\zs ton的课题组\zs 为了证明\zs 深度学习的能力,首次参\zs 加ImageNet图像\zs 识别比赛,其构建\zs 的CNN网络
  AlexNet在ImageNet评测上将错误率从26\%降低到15\%,在第二名(SVM方法)面前以压倒性的优势夺得
  冠军。
  % CNN也正是由于该比赛吸引到了众多研究者的注意。AlexNet的创新点在于:

  % \begin{enumerate}[(1)]
  %   \dawu\item 采用ReLU激活函数,大幅增加收敛速度并且从根本上解决了梯度消失的问题。
  %   \dawu\item 抛弃了“预训练+微调”的方法,完全采用有监督训练方式。同时也影响了深度学习主流学习方法
  %     使其因此变为了纯粹的有监督学习。
  %   \dawu\item 扩展了LeNet5的结构,添加Dropout层减小过拟合,LRN(局部影响归一化)层增强泛化能力并减小过拟合。
  %   \dawu\item 第一次使用GPU加速模型计算。
  % \end{enumerate}

  2016\zs 年3月,由谷\zs 歌(Googl\zs e)公司开发的Alp\zs haGo(基于深度\zs 学习算法)与围棋世\zs 界冠军、职业
  九段\zs 棋手李世石进行围\zs 棋大战,以4比\zs 1的总比分获胜;2016年\zs 末至2017年初,该程序\zs 在中国棋类网站上
  注册\zs 名为“大师”(Ma\zs ster)的帐号并与中日\zs 韩数十位围棋高\zs 手进行快棋对决,连赢60局,\zs 无一败绩;20\zs 17年5月,
  在乌\zs 镇围棋峰会上,Alph\zs aGo与世界排\zs 名第一的围棋冠军\zs 柯洁对战,以3比0\zs 的最终比分获胜。围棋\zs 界公认
  阿尔\zs 法围棋的能力已\zs 经超过人类职业围棋顶\zs 尖水平。

  深度学习虽然已经成为众多科研领域的热门研究内容,但是目前还处于发展阶段,不管是理论方面还是实践
  方面都还有许多问题待解决,不过由于我们处在了一个“大数据”时代,以及计算机处理能力的大大提升,
  新理论的验证周期会大大缩短,人工智能的发展必然会很大程度地改变这个世界。
