\subsection{\hei\xiaosan\textbf{农作物病害识别的国内外研究现状}}
  \subsubsection{\hei\sihao\textbf{国内研究现状}}
    国内在农作物病害识别方面的研究起步较晚,多数是在国外相关研究启发下开始的。
    近年来机器学习领域的研究呈现井喷式发展,随之带动了农作物病害识别的研究,
    并取得了许多卓有成效的识别技术。
    2009年,王守志等实现了基于核K-均值聚类方法的玉米叶部病害识别,实验涉及的4
    种玉米病害识别准确率达82.5\% \upcite{王守志2009基于核};
    2011年,陈丽等提出了一种基于图像处理技术和概率神经网络技术的玉米叶部病害识
    别方法,利用遗传算法优化选择出4个分类能力强的分类特征,由概率网络(PNN)分
    类器识别病害,平均准确率为90.4\%,高于BP神经网络\upcite{陈丽2011概率神经网络在玉米叶部病害识别中的应用};
    % 张建华等提出了一种基于粗糙集和BP神经网络的棉花病害识别方法,
    % 该方法能准确识别4种棉花病害,平均识别准确率达到92.72\% \upcite{张建华2012基于粗糙集和};
    2012年,王树文等利用基本图像处理方法对黄瓜叶部病害图像进行处理,综合运用二次分割、
    形态学滤波得到病斑区域。提取三种特征并采用BP算法训练多层前向神经网络对
    黄瓜病害进行分类,该系统的平均识别准确度为95.31\% \upcite{王树文2012基于图像处理技术的黄瓜叶片病害识别诊断系统研究};
    2013年,张飞云等利用量子神经网络进行玉米病害分类识别,对玉米灰斑病、玉米普
    通锈病和玉米小斑病的识别准确率达92.5\%、97.5\%和92.5\%,高于误差反向传播
    神经网络法的识别率(分别为90.0\%、90.0\%和92.5\%),可用于玉米叶部病害识
    别\upcite{张飞云2013基于量子神经网络和组合特征参数的玉米叶部病害识别}。
    % 2014年,余秀丽等设计并实现了一种基于SVM(Support Vector Machine,支持向量机)
    % 的小麦叶部常见病害识别方法。随机试验结果表明,利用所提取的特征可以有效地实
    % 现小麦叶部常见病害的识别,基于形状特征综合识别率可达99.33\%,利用支持向量
    % 机算法进行小麦病害叶片识别是有效的、可行的\upcite{余秀丽2014基于};
    % 2018年,张航等提出了一种基于卷积神经网络的小麦病害识别方法,利用随机梯度
    % 下降法对一个具有五层结构的网络模型行学习过程控制,其综合识别率可达99\% 
    % \upcite{张航2018一种基于卷积神经网络的小麦病害识别方法}。

  \subsubsection{\hei\sihao\textbf{国外研究现状}}
    国外关于农作物病害识别的研究起步较早,在上世纪九十年代已有多种识别方法被提出。
    在早期的处理方法大多是对病害图像进行前期处理,包括图像分割、滤波、简单分类等
    技术,或是辅助以人工分析,然后再结合农作物病理学知识对处理后的数据进行分类识别。
    1997年,Tucker与Chakraborty提供了一种专用软件,可以检测向日葵和燕麦叶片上的病
    变,提供病变数量和类型以及病害严重程度的数据,但是该软件的分类准确率达不到预想
    程度 \upcite{tucker1997quantitative};
    1999年,Sasaki等使用光谱反射特性和滤光片图像构建了一个植物病害自动诊断系统,
    发现500、600和650mm的滤光镜图像比其他滤光镜图像更适合于识别,在创建鉴别参数对
    健康叶片和患病叶片进行分类后,达到了5\%或更小的误差率\upcite{sasaki1999automatic}。

    由于图像处理技术的限制,早期的农作物病害识别技术并不能满足人们的要求。近年来,
    随着图像处理和人工智能的发展,机器学习应用到农作物病害识别领域的例子越来越多。
    2008年,Phadikar和Sil等介绍了一种基于水稻植株感染图像的水稻病害检测系统,该
    系统使用图像生长、分割技术处理感染部分,然后使用SOM神经网络将四种叶子的感染部
    分进行分类处理,实验结果令人满意\upcite{phadikar2008rice};
    2014年,Mathura与Uttar Pradesh介绍了一种基于邻近像素点像素强度的改进和差直方
    图,与梯度滤波器相配合使用可以对苹果病害的检测达到99\%的准确率 \upcite{dubey2014fruit};
    2016年,Sladojevic等实现了最新一代的卷积神经网络(CNN),该模型能够识别健康
    叶片中的13种不同类型的植物病害,实验结果的平均精确度达到了96.3\% \upcite{sladojevic2016deep};
    2017年,Fuentes和Yoon等提出了一种深度学习方法来检测番茄病害,该方法含有三种
    神经元架构:基于区域的快速卷积网络(Faster R-CNN)、基于区域的全卷积网络
    (R-FCN)和单发多核检测器(SSD),实验结果表明,该系统能有效识别九种不同类型
    的病虫害\upcite{fuentes2017robust}。

\subsection{\hei\xiaosan\textbf{深度学习与神经网络的发展}}
  深度学习是机器学习领域中一种以人工神经网络为架构,根据数据进行表征学习的算法。
  它的前身是人工神经网络,基本特点是模仿人脑神经元处理和传递信息的方式,本质是解决贡献度的分配问题。

  对深度学习的研究最早可以追溯到1943年,数学家皮兹(W.Pitts)和神经科学家麦卡洛克
  (W.S.McCulloch)建立了一个基于神经网络和数学的模型,称为MCP模型。该模型是按照人脑神经元的结构和工作
  原理构造出来的一个抽象的、简化了的模型,也被人们称为“模拟大脑”\upcite{mcculloch1990logical}。
  1958年,计算机科学家罗森布拉特(Rosenblatt)提出了感知器(Perceptrons),
  这是一个由两层神经元组成的神经网络。这也是第一次将MCP模型用于机器学习分类。
  “感知器”算法使用MCP模型对输入的多个维度数据分为两类,且能够在训练过程中中自
  动优化并更新权值。1962年,Novikoff定理证明该方法具有收敛性,理论与实践效果引起第一次神经网络热潮。

  神经网络之父Geoffery Hinton于1986年发明了适用于多层感知器(MLP)的BP(Backpropagation)
  算法,该方法采用Sigmoid进行非线性映射,有效解决了非线性分类和学习的问题。
  Sigmoid函数的输出在(0,1)内,单调连续且易于求导,非常适合用作输出层。
  但是它也有软饱和性的缺点,一旦落入饱和区,$f'(x)$就会变得接近于0,很容易产生梯度消失。
  90年代中期,支持向量机(Support Vector Machine,SVM)算法诞生,随之各种浅层机器学习
  模型被提出。支持向量机又名支持向量网络,是在分类与回归分析中分析数据的监督式学习模型与相关的学习算法\upcite{cortes1995support}
  支持向量机与其他浅层机器学习算法的提出再次使深度学习的发展受到阻碍。
  % 2006年Geoffrey Hinton和Ruslan Salakhutdinov提出了深层网训练中梯度消失的解决方案:
  % 无监督预训练下对权值进行初始化,有监督训练权值微调\upcite{hinton2006reducing}。2011年,ReLU
  % 激活函数被提出,该激活函数能够有效的抑制梯度消失问题。

  % 2011年,微软首次将深度学习应用在语音识别领域,取得了重大突破。微软研究院和Google的语音识别
  % 研究人员先后采用卷积神经网络使语音识别错误率降低20%\textasciitilde30%,是语音识别领域十多年来的突破性进展。
  Hinton所带领的课题组为了测试深度学习模型的能力,于2012年使用首次使用CNN网络AlexNet参加ImageNet
  图像识别比赛,在ImageNet评测上将错误率从26\%降低到15\%,在第二名(SVM方法)面前以压倒性的优势夺得
  冠军。
  % CNN也正是由于该比赛吸引到了众多研究者的注意。AlexNet的创新点在于:

  % \begin{enumerate}[(1)]
  %   \dawu\item 采用ReLU激活函数,大幅增加收敛速度并且从根本上解决了梯度消失的问题。
  %   \dawu\item 抛弃了“预训练+微调”的方法,完全采用有监督训练方式。同时也影响了深度学习主流学习方法
  %     使其因此变为了纯粹的有监督学习。
  %   \dawu\item 扩展了LeNet5的结构,添加Dropout层减小过拟合,LRN(局部影响归一化)层增强泛化能力并减小过拟合。
  %   \dawu\item 第一次使用GPU加速模型计算。
  % \end{enumerate}

  % 2016年3月,由谷歌(Google)公司开发的AlphaGo(基于深度学习算法)与围棋世界冠军、职业
  % 九段棋手李世石进行围棋大战,以4比1的总比分获胜;2016年末至2017年初,该程序在中国棋类网站上
  % 注册名为“大师”(Master)的帐号并与中日韩数十位围棋高手进行快棋对决,连赢60局,无一败绩;2017年5月,
  % 在乌镇围棋峰会上,AlphaGo与世界排名第一的围棋冠军柯洁对战,以3比0的最终比分获胜。围棋界公认
  % 阿尔法围棋的能力已经超过人类职业围棋顶尖水平。

  深度学习虽然已经成为众多科研领域的热门研究内容,但是目前还处于发展阶段,不管是理论方面还是实践
  方面都还有许多问题待解决,不过由于我们处在了一个“大数据”时代,以及计算机处理能力的大大提升,
  新理论的验证周期会大大缩短,人工智能的发展必然会很大程度地改变这个世界。

  \subsection{\hei\xiaosan\textbf{论文结构和研究内容}}
  \subsubsection{\hei\sihao\textbf{主要研究内容}}
    本文以提高卷积网络在小麦病害图像识别的准确率为目标,提出了具有12层结构的卷积网络。
    针对前期实验的不足之处对网络结构加以更改,在得到了训练过程中表现良好的网络模型的模型之后,
    将其与经典卷积网络模型LeNet-5、AlexNet的实验结果进行比较,分析本文模型的长处与不足,验证
    该模型在小麦病害图像识别中的有效性。
    

  \subsubsection{\hei\sihao\textbf{论文结构}}
    第一章是引言,首先介绍了小麦病害图像识别的研究背景和意义,
    然后综述了深度学习与神经网络的发展历史及研究现状,
    最后介绍了本文的主要研究内容和章节安排。

    第二章主要介绍了卷积的基础理论知识,然后简单介绍了卷积神经网络的发展及其特点和主要结构层次。

    第三章搭建了一个结构上较新颖的卷积神经网络模型,并针对小麦病害识别加以修改,
    使其尽量符合预期要求。接下来主要介绍了实验内容,首先是数据源的获取和处理,然后是不同模型
    在相同数据源下的训练过程。

    第四章是针对该实验结果的分析以及对不同模型的结果加以对比,说明本文新模型的优势。
    然后对数据集的质量进行了解释,以及数据集对训练结果的影响。最后是对本文的总结分析。
