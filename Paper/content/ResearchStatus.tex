\subsection{\textbf{国内外研究现状}}
\subsubsection{\textbf{国内研究现状}}
国内将深度学习应用在图像处理尤其是农作物病害识别方面的研究起步较晚,大多数
在人工智能浪潮下开始此领域研究的,但是近年来该领域的研究呈现井喷式发展,并
取得了许多卓有成效的识别技术。
2009年,王守志等(王守志等,2009)实现了基于核K-均值聚类方法的玉米叶部病害识别,
实验涉及的4中玉米病害识别准确率达82.5\% \cite{王守志2009基于核};
2011年,陈丽等(陈丽等,2011)提出了一种基于图像处理技术和概率神经网络技术的玉米
叶部病害识别方法,利用遗传算法优化选择出4个分类能力洽购的分类特征,由概率网络(PNN)
分类器识别病害,平均识别准确率为90.4\%,高于BP神经网络\cite{陈丽2011概率神经网络在玉米叶部病害识别中的应用};
2012年,张建华等(张建华等,2012)提出了一种在自然条件下基于粗糙集和BP神经网络的
棉花病害识别方法,准确识别了4种棉花病害,平均识别准确率达到92.72\% \cite{张建华2012基于粗糙集和};
王树文等(王树文等,2012)利用基本图像处理方法对黄瓜叶部病害图像进行处理,综合运用
二次分割、形态学滤波得到病斑区域。提取三种特征并采用BP算法训练多层前向人工神经网络
对黄瓜病害进行分类,检测系统的黄瓜叶部病害平均识别精度为95.31\% \cite{王树文2012基于图像处理技术的黄瓜叶片病害识别诊断系统研究};
2013年,张飞云等(张飞云等,2013)利用量子神经网络进行玉米病害分类识别,对玉米灰斑病、
玉米普通锈病和玉米小斑病的识别准确率达92.5\%、97.5\%和92.5\%,高于误差反向传播神经网
络法的识别率(分别为90.0\%、90.0\%和92.5\%),可用于玉米叶部病害识别\cite{张飞云2013基于量子神经网络和组合特征参数的玉米叶部病害识别};
2014年,余秀丽等(余秀丽等,2014)设计并实现了一种基于SVM(Support Vector Machine)
的小麦叶部常见病害识别方法。随机试验结果表明,利用所提取的特征可以有效地实现小麦叶部常
见病害的识别,基于形状特征综合识别率可达99.33\%,利用支持向量机算法进行小麦病害叶片识
别是有效的、可行的\cite{余秀丽2014基于};
2015年,谭文学等(谭文学等,2015)设计了深度学习神经网络的果蔬果体图像识别方法,基于
对网络误差的传播分析,提出弹性动量的参数学习方法,以苹果为例进行果体病理图像的识别试验。
结果表明,该方法召回率为98.4\% \cite{谭文学2015基于弹性动量深度学习神经网络的果体病理图像识别};
2018年,张航等(张航等,2018)提出了一种基于卷积神经网络的小麦病害识别方法,构建一个
具有五层结构的深度学习模型并利用随机梯度下降法进行学习过程控制,其综合识别率可达99\% \cite{张航2018一种基于卷积神经网络的小麦病害识别方法};


\subsubsection{\textbf{国外研究现状}}

