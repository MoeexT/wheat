\subsection{\hei\xiaosan\textbf{论文结构和研究内容}}
  \subsubsection{\hei\sihao\textbf{主要研究内容}}
    本文以提高卷积网络在小麦病害图像识别的准确率为目标,提出了具有12层结构的卷积网络。
    针对前期实验的不足之处对网络结构加以更改,在得到了训练过程中表现良好的网络模型的模型之后,
    将其与经典卷积网络模型LeNet-5、AlexNet的实验结果进行比较,分析本文模型的长处与不足,验证
    该模型在小麦病害图像识别中的有效性。
    

  \subsubsection{\hei\sihao\textbf{论文结构}}
    第一章是引言,首先介绍了小麦病害图像识别的研究背景和意义,
    然后综述了深度学习与\zs 神经网络的\zs 发展历史及\zs 研究现状,
    最后介绍\zs 了本文的主要研究内容和章节安排。

    第二章主要介绍了卷积的基础理论知识,然后简单介绍了卷积神经网络的发展及其特点和主要结构层次。

    第三章搭建了一个结构上较新颖的卷积神经网络模型,并针对小麦病害识别加以修改,
    使其尽量符合预期要求。接下来主要介绍了实验内容,首先是数据源的获取和处理,然后是不同模型
    在相同数据源下的训练过程。

    第四章是针对该实验结果的分析以及对不同模型的结果加以对比,说明本文新模型的优势。
    然后对数据集的质量进行了解释,以及数据集对训练结果的影响。最后是对本文的总结分析。